\section{Aggregate-Level Replication}
In this section, I compute the nominal and real interest rates implied by the Euler equation using the VAR estimated from the full sample (1960:I to 2015:II) of the aggregate series described in \autoref{aggregate-data}. As in \cite{collard11}, I take the discount rate $\beta$ to be 0.9926 (so that households discount at an annual rate of 3 percent) and the coefficient of risk aversion $\alpha$ to be 2. I look at four specifications of utility:

\textit{SEP:} These are standard CRRA preferences \eqref{crra-utility}, in which there is no habit formation ($\phi = 0$) and consumption and leisure are assumed to be additively separable ($\nu = 1$).

\textit{SEP + HP:} Setting $\phi = 0.8$ gives habit formation as in \cite{fuhrer00}. Now period utility depends on the ratio of current to previous period consumption, while leisure is still assumed to be separable ($\nu = 1$): $$u(C_t, C_{t-1}) = \frac{(C_t/C_{t-1}^\phi)^{1-\alpha}}{1-\alpha}$$

\textit{NSEP:} On the other hand, nonseparability in consumption and leisure is introduced by letting $\nu = 0.34$, which implies a work share of 30 percent in the absence of habit formation ($\phi = 0$): $$u(C_t, l_t) = \frac{(C_t^\nu l_t^{1-\nu})^{1-\alpha}}{1-\alpha}$$

\textit{NSEP + HP:} Finally, letting both $\phi = 0.8$ and $\nu = 0.34$ gives the full \cite{collard11} model in \eqref{collard-utility} with both habit formation and nonseparability in consumption and leisure.



\subsection{Correlation of implied and observed rates}
Summary statistics for the implied rates under each specification are reported below in \autoref{implied-vs-ffr-nipa}, as well as the correlation between each implied rate and the effective federal funds rate.

\begin{table}[t]
\centering
\caption{Summary statistics for nominal and real rates (annualized rates)}
\label{implied-vs-ffr-nipa}
\begin{tabular}{lccccc} \hline
& Data & SEP & SEP + HP & NSEP & NSEP + HP \\ \hline
\multicolumn{6}{c}{Real interest rates} \\ \hline
\csvreader[head to column names, late after line = \\]%
  {tables/nipa-real.csv}{}%
  {\stat & \data & \sep & \sephp & \nsep & \nsephp} \hline
\multicolumn{6}{c}{Nominal interest rates} \\ \hline
\csvreader[head to column names, late after line = \\]%
  {tables/nipa-nominal.csv}{}%
  {\stat & \data & \sep & \sephp & \nsep & \nsephp} \hline
\end{tabular}
\end{table}

The first thing that stands out is the presence of strong positive correlations overall, but especially for nominal rates in the specifications withough habit formation. Notably, the correlation between the nominal rate implied by CRRA preferences (SEP) and the historical nominal effective federal funds rate is 0.52, while adding nonseparability in leisure (NSEP) increases this correlation to 0.707. In the utility models with habit formation (SEP + HP and NSEP + HP), the correlation is less positive for nominal rates and essentially zero for real rates.

These strong positive correlations are noticeably higher than the still-positive correlations found by \cite{collard11}, to say nothing of the strongly negative values found by \cite{canzoneri07}. As a check, I reestimate the VAR and recompute the implied rates and correlations using only the time period spanned by \cite{collard11}, stopping at 2006:IV instead of 2015:II. The correlations for nominal rates from this restricted sample more closely resemble their results, though the ones for real rates are still rather different. I summarize the restricted sample results in \autoref{implied-vs-ffr-nipa-collard} in the appendix. In \autoref{correlation-comparison-nipa}, I compare the full and restricted sample correlations to those found in the other two papers. Note that \cite{canzoneri07} paper examines several utility specifications, including CRRA (SEP) and Fuhrer habit preferences (SEP + HP), but does not include analysis of nonseparability in leisure.

\begin{table}[t]
\centering
\caption{Comparison of correlations between implied rates and effective FFR}
\label{correlation-comparison-nipa}
\begin{tabular}{lcccccc} \hline
                   & SEP   & SEP + HP & NSEP  & NSEP + HP & Start  & End \\ \hline
\multicolumn{7}{c}{Real interest rate correlation} \\ \hline
Full Sample        & 0.197 & -0.050   & 0.261 & 0.033     & 1960:I & 2015:II \\
Restricted Sample  & 0.020 & -0.098   & 0.065 & -0.058    & 1960:I & 2006:IV \\
\cite{collard11}   & 0.05  & 0.15     & 0.28  & 0.27      & 1960:I & 2006:IV \\
\cite{canzoneri07} & -0.37 & -0.07    & ---   & ---       & 1966:I & 2003:IV \\ \hline
\multicolumn{7}{c}{Nominal interest rate correlation} \\ \hline
Full Sample        & 0.520 & 0.142    & 0.707 & 0.390     & 1960:I & 2015:II \\
Restricted Sample  & 0.255 & 0.030    & 0.563 & -0.225    & 1960:I & 2006:IV \\
\cite{collard11}   & 0.26  & 0.04     & 0.63  & 0.38      & 1960:I & 2006:IV \\
\cite{canzoneri07} & 0.20  & -0.10    & ---   & ---       & 1966:I & 2003:IV \\ \hline
\end{tabular}
\end{table}

The extreme variation in correlations found suggests two points at which this analysis is not sufficiently robust.

First, comparing the correlations computed from the full sample to those from the restricted sample highlights the impact of the inclusion of the additional quarters from 2007:I to 2015:II. Scatter plots for both samples are shown in \autoref{nipa-full-vs-restricted-scatter}. (In particular, the data points in the restricted sample plot are not a subset of those in the full sample plot because the implied rates for each were computed using different VAR estimates.) The difference between the two samples is of course the era of near-zero interest rates following the Great Recession in 2008, which can be seen in the full sample plot as the cluster of observations on the FFR = 0 line. These, along with the outliers in the bottom left (which are also at the zero lower bound), drive the more strongly positive correlation in the full sample.

\begin{figure}[t]
\centering
\captionsetup{singlelinecheck=false, justification=centering}
\caption{SEP implied vs. observed nominal rates \\ Left: full sample, $\rho = 0.520$. Right: restricted sample, $\rho = 0.255$.}
\label{nipa-full-vs-restricted-scatter}
\begin{tabular}{cc}
\includegraphics[width=0.5\textwidth]{figs/nipa/nominal_sep_scatter} &
\includegraphics[width=0.5\textwidth]{figs/nipa/nominal_sep_scatter_collard} \\
\end{tabular}
\end{figure}

Even within the same time span, comparing the restricted sample correlations to those of \cite{collard11} highlights the fragility of these results with respect to small changes in methodology. I follow the specifications in \cite{collard11} as closely as possible, except where it is not possible or not completely clear what they did. As mentioned in the previous section, due to lack of availability of data, I generate aggregate real consumption from the chain quantity indices scaled by the 2009 nominal consumption, while they use real consumption directly. I also estimate the VARs using log of gross quarterly inflation and interest rates $\pi_t$ and $ffr_t$, while it is possible that \cite{collard11} may have used annualized rates and/or scaled them to units of percentage points. Other possible differences include our choices of base year (2009 in my analysis, versus 2000) and whether we take the natural log of real dollars (as I do) or billions of real dollars.

All of this is to say that the correlation between the implied and observed rates is probably not the most reliable metric by which we should judge the fit of the consumption Euler equation to the data --- even though it's arguably the focus of both of these previous papers. This is particularly evident upon qualitatively examining the paths of the observed and implied rates, which are shown for the SEP case in \autoref{implied-vs-ffr-nipa-sep} below. (Plots of the other three utility specifications are found beginning with \autoref{implied-vs-ffr-nipa-others} in the appendix.)

\begin{figure}[h]
\centering
\captionsetup{singlelinecheck=false, justification=centering}
\caption{SEP implied vs. observed rates \\ Left: real rates, $\rho = 0.197$. Right: nominal rates, $\rho = 0.520$.}
\label{implied-vs-ffr-nipa-sep}
\begin{tabular}{cc}
\includegraphics[width=0.5\textwidth]{figs/nipa/real_sep} &
\includegraphics[width=0.5\textwidth]{figs/nipa/nominal_sep}
\end{tabular}
\end{figure}

Despite the positive correlations of 0.197 and 0.520 respectively, both the real and nominal implied rates persist in moving in the opposite direction from the respective ex post rates. The paths of the implied rates look very similar to those found by \cite{canzoneri07}, who found them to be negatively correlated with the observed rates. A more consistent metric of fit is the correlation of the spread with the stance of monetary policy, which I discuss next.

% TODO: My results also differ from Collard and Canzoneri in volatility in SEP + HP
