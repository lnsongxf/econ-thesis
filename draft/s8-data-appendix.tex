\section{Data appendix}

\subsection{Aggregate-level data}
\label{aggregate-data-appendix}

First, I describe the construction of the endogenous variables (prior to taking logs) used in the aggregate-level replication.

\textit{Per capita real consumption} $C_t$: Aggregate real consumption is defined as the sum of the chain quantity indices (2009 = 100) for personal consumption expenditures on nondurable goods (\texttt{DNDGRA3Q086SBEA}) and services (\texttt{DSERRA3Q086SBEA}), all multiplied by the sum of nominal nondurables (\texttt{PCEND}) and services (\texttt{PCESV}) consumption in 2009\footnote{I generate aggregate real consumption from the chain quantity indices because the real consumption variables used by Collard and Dellas, \texttt{PCNDGC96} and \texttt{PCESVC96}, were not available from FRED for the quarters before 1999:I.}. This amount is divided by the civilian noninstitutional population (\texttt{CNP16OV}) to get per capita real consumption.

\textit{Gross quarterly inflation} $\Pi_t$: In each quarter, the implicit price deflator $P_t$ is calculated by dividing aggregate nominal consumption (\texttt{PCEND + PCESV}) by aggregate real consumption (described above). Then gross quarterly inflation is defined as the growth rate of the deflator: $\Pi_t = \frac{P_t}{P_{t-1}}$.

\textit{Leisure fraction} $l_t$: Labor fraction $h_t$ is defined as the average weekly hours worked in the nonfarm business sector (\texttt{PRS85006023}), multiplied by the civilian employment-to-population ratio (\texttt{EMRATIO}). Following Collard and Dellas, I then rescale so that the mean over all quarters is $\frac{1}{3}$, corresponding to an average of 8 hours worked per weekday. Then the leisure fraction is given by $l_t = 1 - h_t$.

\textit{Per capita real disposable income} $RDI_t$: This is computed by dividing real disposable income (\texttt{DPIC96}) by the civilian non-institutional population.

\textit{Per capita real output less consumption} $YMC_t$: Defined as real gross domestic product (\texttt{GDPC96}) minus aggregate real consumption, again divided by the civilian noninstitutional population.

\textit{Gross quarterly effective federal funds rate} $FFR_t$: This is computed by raising the gross annualized rate (\texttt{DFF}) to the one-fourth power.

\textit{Continuous Commodity Index} $CCI_t$: I use the CCI ending price on the first day of each quarter, obtained from Bloomberg. As mentioned, the CCI is the continuation of the CRB Index used by \cite{canzoneri07} and Collard and Dellas. What is called the CRB Index today is calculated slightly differently and exists only since 1995.



\subsection{Household-level data}
\label{household-data-appendix}

To construct equivalent series from the CEX data, I first generate the following observation-level (nominal) variables:

\textit{Consumption:} Following \cite{heathcote10}, I define consumption of nondurable goods and services as the sum of the following expenditure categories: food and beverages (\texttt{FOOD} $+$ \texttt{ALCBEV}), clothing (\texttt{APPAR}), gasoline (\texttt{GASMO}), household operation (\texttt{HOUSOP}), public transportation (\texttt{PUBTRA}), medical care excluding health insurance (\texttt{HEALTH} $-$ \texttt{HEALTHIN}), recreation (\texttt{ENTERT}), tobacco (\texttt{TOBACC}), and education (\texttt{READ} $+$ \texttt{EDUCA}).

\textit{Hours worked:} I use the weekly hours worked by the household's reference person (\texttt{INC\_HRS1}). The reference person is the first person mentioned by the survey respondent when asked to ``Start with the name of the person or one of the persons who owns or rents the home.''

\textit{Disposable income:}  I use after-tax income (\texttt{FINCATAX}), as in \cite{krueger15}.

\textit{Output less consumption:} Defined as before-tax income (\texttt{FINCBTAX}) minus consumption (defined above).

Consumption, disposable income, and output less consumption are each deflated by the unadjusted Consumer Price Index for nondurables for urban consumers (\texttt{CUUR0000SAN} in FRED), following Vissing-Jorgensen, rescaled to 2009 dollars to correspond with the aggregate-level data. The expenditure categories included in consumption were chosen to allow for the possibility of deflating each category by its own CPI (for example, \texttt{CPIFABNS} from FRED for food and beverages). However, the result of doing so was found to differ only negligibly from using a single CPI.

The CEX provides population weights for each household, which are calibrated so that summing the population weights in a given quarter approximates the number of households in the United States that quarter, while taking the weighted sum of the number of household members approximates the total population. I take the weighted mean of hours worked for each quarter and use it to generate labor fraction $l_t$ as in the previous section. For each of consumption, disposable income, and output less consumption, I take the weighted sum each quarter and divide it by the population to get per capita variables $C_t$, $RDI_t$, and $YMC_t$.

Finally, I seasonally adjust log consumption $c_t$ by regressing it on indicators of the quarters and subtracting off the non-first quarter coefficients.