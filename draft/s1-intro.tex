\section{Introduction}
Perhaps the main criticism of modern macroeconomic models (in particular, DSGE models) is that the microfoundational assumptions on which they're based often don't actually fit the data very well. \cite{smith14} singles out the consumption Euler equation, which expresses intertemporal consumption choice in terms of the real interest rate $r_t$. In its typical form: $$\frac{1}{1 + r_t} = \beta \E_t \bracket{ \frac{\d U_t / \d C_{t+1}}{\d U_t / \d C_t} }$$

\cite{canzoneri07} compute the interest rate implied by the consumption Euler equation under several utility specifications. They find that their computed rates are actually negatively correlated with historical money market rates, and furthermore that the spread is correlated with the stance of monetary policy. These results are potentially extremely damaging to the validity of macroeconomic models which assume the Euler equation implied rate and the actual interest rate to be the same -- that is, nearly all macro models. \cite{collard11} repeat this exercise, adding utility nonseparable in consumption and labor, and in fact find the looked-for positive correlation with observed rates.

In this paper, I first attempt to replicate the findings of Canzoneri, Cumby, and Diba (henceforth abbreviated CCD) and \cite{collard11} using new data up through the second quarter of 2015.  This portion includes computing Euler equation implied rates and correlating the spread between implied and observed rates with the stance of monetary policy. The consumption and income data for this section are all national aggregates from the National Income and Product Accounts (NIPA).

The main novel contribution of this paper is the introduction of limited asset market participation to the implied rate framework, inspired by \cite{vissing02}. Specifically, I aggregate household-level data from the Consumer Expenditure Survey (CEX) for bondholders and nonbondholders. I perform the same analyses on the time series of these two groups to test the hypothesis that interest rates implied by bondholders' consumption paths will more resemble observed rates than those from nonbondholders. The intuition for this idea is clear: we expect households with positions in the bond market to adjust their consumption in response to changes in the interest rate, while we don't expect nonbondholders to do so.

% TODO: add results