\section{Literature}
There is a substantial body of literature on how the standard representative agent model and its resultant Euler equations fails to fit the data in various ways. \cite{parker99} points to several potential explanations for these failures. Among the possibilities: preferences may be nonseparable, aggregation across heterogeneous households may produce bias, or actual asset markets may not be complete. In this paper, I focus on the first two explanations.



\subsection{Nonseparability in utility models}
Standard preferences feature constant relative risk aversion (CRRA utility), which assume additive separability both across time and between consumption and labor. Relaxing these simplifying assumptions has been shown to explain some results from the empirical literature which initially seemed incompatible with household optimizing behavior.

\cite{fuhrer00} develops a utility model with habit formation in order to explain a key feature of aggregate data: the ``hump-shaped'' responses of consumption and inflation, which he demonstrates aren't seen in impulse response functions computed using the standard CRRA model. In addition to being intuitively reasonable, his addition of habit formation (equivalently, nonseparability across time) allows the model to explain significant delays in response to monetary policy shocks.

\cite{basu02} observe that assuming additive separability between consumption and labor\footnote{This is traditionally carried out implicitly when utility is made a function of consumption alone and labor only comes into the optimization as a source of income.} leads to estimations that the income effect of a permanent wage increase strongly overpower the substitution effect, hence reducing labor supply. This implication is not borne out in empirical comparisons of income versus hours worked. Using aggregate data, they show that the King-Plosser-Rebelo utility model (which is nonseparable in consumption and leisure) leads to more realistic estimates of the intertemporal elasticity of substitution for consumption in the 1980s and 1990s, though not necessarily as well for earlier data.



\subsection{Euler equation implied rates}
Another particular strand of the literature on the failure of the Euler equation stems from \cite{canzoneri07}. They computed the interest rates implied by the Euler equation and compare them to observed historical money market rates. The two series turn out to be negatively correlated, a result which is robust to the addition of several different habit formation specifications to the standard CRRA utility model. Furthermore, they find that the spread between the observed rate and the Euler equation implied rate is correlated with the stance of monetary policy, presenting a challenge to modern macroeconomic models which equate the two rates. Finally, they compute impulse response functions of the implied rate to a monetary shock, which move nearly in a mirror image to the observed rate.

This paper is followed by \cite{collard11}, who undertake the same exercise but impose the additional restriction of nonseparability in consumption and leisure, citing the work of \cite{basu02} discussed earlier. Collard and Dellas find that the enforcement of this nonseparability makes the correlation between observed and Euler equation implied rates strongly positive and the difference between their volatilities smaller. However, the actual path of the Euler implied rate still differs substantially from observed rates.

In a similar vein, \cite{gareis13} compute the Euler equation implied rate from the \cite{smets07} DSGE model, which happens to incorporate both habit formation and non-nonseparability in consumption and leisure. They too find a positive correlation between actual and implied real interest rates. However, their Euler equation implied rates are much more volatile than (about five times the standard deviation of) the actual rates. They also estimate the true distribution of correlations by using the Smets and Wouters model as a data-generating process for a Monte Carlo experiment. They simulate 1000 time series of consumption, inflation, hours worked, and interest rates and correlate the ``actual'' and Euler implied rates. The resulting distribution of correlations is centered around zero (though the median is slightly positive), in contrast to their non-simulated findings.



\subsection{Heterogeneous agents}
Perhaps the most obvious argument against the representative agent model is that in practice, the real population of consumers is not at all homogeneous. Another subset of the literature shows how introducing heterogeneity to the representative agent model can reconcile theory with ostensible empirical irregularities.

\cite{campbell89} examine heterogeneity in optimizing behavior. Specifically, on top of the standard optimizing agent who consumes his permanent income, they posit an additional class of ``rule of thumb'' consumers who do not optimize and consume exactly their income each period. They interpret the coefficient of disposable income on consumption as the fraction of such current-income consumers, which is estimated to be about or above 50\%, depending on the specification of controls. They conclude that their model with rule of thumb consumers included better fits some stylized facts found in aggregate data, including consumption's response to income but not real interest rate changes, as well as the ``excess smoothness'' of consumption compared to what is predicted by the permanent income hypothesis model.

A related approach is taken by \cite{guvenen06}, who seeks to reconcile macroeconomic assumptions of the elasticity of intertemporal substitution (near 1) with empirical estimates (near 0). He proposes that this inconsistency is a consequence of dissimilarity of high-elasticity stockholders and low-elasticity non-stockholders. Despite making up a relatively small fraction of the overall population, stockholders hold most of the wealth, so analysis of savings and investment typically reflect their dynamics. On the other hand, consumption is much more evenly distributed, so aggregate consumption data reflect the low elasticity of the majority --- that is, the poor. 

\cite{vissing02} also looks at heterogeneity of asset market participation as a means of reconciling differing estimates of the EIS. Using household-level data from the CEX, she classifies households as being stockholders, bondholders, and non-assetholders. She then estimates the EIS from a log-linearized Euler equation separately for stock- and bondholders using using, respectively, returns on the New York Stock Exchange composite and returns on Treasury bills. She argues that the Euler equation should only hold for a household with a position in that particular asset, and that the inclusion of non-assetholders in Euler equation estimations distorts the results. I take this idea and invert Vissing-Jorgensen's methodology --- rather than estimating the EIS from the Euler equation using historical returns, I compute the interest rates implied by the Euler equation assuming some value for the EIS.
