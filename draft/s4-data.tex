\section{Data}

\subsection{Aggregate-level data}
\label{aggregate-data}

In the aggregate-level analysis, the endogenous variables making up $y_t$ in the VAR model are constructed according to \cite{collard11} whenever possible. Except where mentioned, all the raw time series used are obtained from the St. Louis Fed's Federal Reserve Economic Data (FRED), with variable names in parentheses. Data are at the quarterly level and seasonally adjusted when appropriate, spanning 222 quarters from 1960:I to 2015:II, inclusive. Real dollar values are in 2009 dollars. All lowercase variables in the vector $y_t$ denote the natural log of the respective capitalized variable described in the \hyperref[aggregate-data-appendix]{data appendix} \textit{except the leisure fraction $l_t$}.



\subsection{Household-level data}
\label{household-data}

For the comparison of bondholders to nonbondholders, I reuse the inflation, federal funds rate, and CCI variables constructed in the previous section. I generate separate time series for the other four endogenous variables for both bondholders and nonbondholders by aggregating household-level data from the Consumer Expenditure Survey from 1996:I to 2012:IV (68 quarters). This process is also described in more detail in the \hyperref[household-data-appendix]{data appendix}.

The CEX is a rotating panel of representative ``consumer units''\footnote{I refer to these consumer units informally as households, though the CEX does actually distinguish between the two terms, allowing for multiple consumer units to dwell in the same physical household. However, it is the consumer unit level at which financial decisions are made and reported to the survey-takers, and hence at which the analysis in this paper is carried out.} in the United States, which are interviewed each quarter for five consecutive quarters. Each observation is a household-quarter. The first interview is for practice, and is not included in the reported survey data. Each quarter, 20 percent of the households rotate out of the survey after their fifth interview, and a new 20 percent rotate in. Households report their expenditures in very detailed categories each quarter. Demographic and income data are collected in the second and fifth interviews, and asset holdings information is collected only in the fifth interview\footnote{The CEX in fact consists of two separate surveys: the Interview Survey, which I have just described, and the Diary Survey, in which households report weekly expenditures on frequently purchased items. I use the Interview Survey exclusively.}.

Following standard practice, I discard the 86,403 observations (out of 478,894) which are flagged by the CEX as being incomplete income respondents (\texttt{RESPSTAT}\footnote{The variable names in the remainder of this section refer to CEX variables unless otherwise specified.} $= 2$). I don't exclude households who are potentially borrowing constrained (indicated by a low wealth-to-income ratio) because I conjecture that credit constraints are a mechanism through which nonbondholders may be less able to optimize their consumption according to the consumption Euler equation.

I determine whether or or not to label each household a bondholder using the criteria set forth by \cite{vissing02}. In the fifth interview, the CEX asks each household to estimate its current holdings in a number of asset categories, as well as how those holdings have changed in the preceding year (four quarters). I use a positive response in the asset categories ``U.S. Savings Bonds'' and ``stocks, mutual funds, private bonds, government bonds, or Treasury notes'' to determine bondholder status, despite that this definition likely creates some false positives, such as households which hold stocks but not bonds. It is difficult to achieve a more complete separation of households.

Either all observations belonging to a particular household are labeled bondholder observations or none of them are. I do not allow for a household's bondholder status to change between interviews. A household is defined to be a bondholder if it had positive holdings of at least one of the two asset categories one year before the asset holdings questions are asked in the fifth interview (i.e. at the time of the first interview) --- specifically, if one of the following holds:
\begin{enumerate}
\item The household reports holding the same amount of the asset as a year ago (\texttt{COMPBND} or \texttt{COMPSEC} $= 1$), and reports a positive current holdings amount \texttt{USBNDX} or \texttt{SECESTX} $> 0$)
\item The household reports lower holdings of the asset than a year ago (\texttt{COMPBND} or \texttt{COMPSEC} $= 2$)
\item The household reports an increase in holdings in the past year (\texttt{COMPBND} or \texttt{COMPSEC} $= 3$) by an amount less than the current holdings (\texttt{COMPBNDX} $<$ \texttt{USBNDX} or \texttt{COMPSECX} $<$ \texttt{SECESTX})
\end{enumerate}

Summary statistics for bondholders and nonbondholders are reported in \autoref{bondholder-nonbondholder-summary}. Since bondholders represent a fairly small fraction of the total sample, I include all bondholder observations in the bondholder aggregate but take a random sample of the nonbondholder observations in order to equalize sample size.

\begin{table}[b]
\centering
\caption{Summary statistics for bondholders and nonbondholders (Per capita, 2009 dollars)}
\label{bondholder-nonbondholder-summary}
\begin{tabular}{lcccc} \hline
& \multicolumn{2}{c}{Bondholders} & \multicolumn{2}{c}{Nonbondholders} \\
& Mean & SD & Mean & SD \\ \hline
Consumption             & 2,326  & 176   & 1,624  & 89.99 \\
Hours worked            & 41.10  & 1.19  & 40.55  & 0.72 \\
Disposable income       & 76,789 & 5,541 & 51,901 & 3,958 \\
Output less consumption & 80,951 & 5,154 & 53,003 & 3,655 \\ \hline
Observations            & \multicolumn{2}{c}{55,847} & \multicolumn{2}{c}{336,344} \\
Households              & \multicolumn{2}{c}{16,959} & \multicolumn{2}{c}{125,895} \\ \hline
\end{tabular}
\end{table}