\section{Conclusion}
My findings at the aggregate level are mostly consistent with the analyses of \cite{canzoneri07} and \cite{collard11}. I show that the paths of the observed and Euler equation implied interest rates diverge following a monetary policy shock, which causes the spread between the two to be systematically related to the stance of monetary policy. Seemingly at odds with this result, I do find strong positive correlations between the implied and observed rates. However, I demonstrate that these correlations are not a robust measurement of the fit of the consumption Euler equation to the data. They are extremely sensitive to the choice of time period, as well as other relatively small methodological changes. Unlike Collard and Dellas, I do not conclude that introducing nonseparability in consumption and leisure to the utility model substantially improves the fit: though the implied rates under nonseparability are correlated more positively with observed rates, there is virtually no effect on the more reliable measure, the response of the spreads to the federal funds rate.

One possible weakness in this methodology is the assumption that the dynamics of the system (including consumption, inflation, income, and the federal funds rate) can be modeled by a vector autoregression. VARs are favored for being simple and model-agnostic, but they may not be sophisticated enough to accurately estimate the conditional moments used in computing the implied rates. A future researcher could try estimating the conditional moments from a larger-scale model such as the \cite{smets07} DSGE model (though this presents an endogeneity problem, as the Smets and Wouters model critically assumes a version of the consumption Euler equation).

At the household level, distinguishing between bondholders and nonbondholders in computing implied rates gives results that qualitatively resemble what one might expect: that bondholders optimize their consumption in response to interest rate changes as the Euler equation predicts, while nonbondholders' behavior is less well described by the Euler equation. Compared to nonbondholders, I find that the interest rates implied by bondholders' consumption paths are more strongly correlated with ex post rates, and the effect of the federal funds rate on the spread, while still significant and negative, is weaker. However, neither the difference in correlations nor the difference in coefficients between bondholders and nonbondholders is statistically significant. This may result in part from the imperfect determination of which households constitute bondholders, which is limited by the questions asked in the CEX. Another potential explanation is the high volatility of consumption growth in the data aggregated from the household level, which may reflect the relatively small sample size for bondholders more than the actual variation experienced by consumers. Further investigation is needed, possibly involving examining additional data or bootstrapping the sample of bondholders to achieve more observations.

These findings continue to pose a challenge to the Euler equation's privileged role in macroeconomic estimation and forecasting. While it is a useful and elegant abstraction, a wealth of empirical evidence has accumulated against its predictions. In particular, my conclusion that the spread between implied and historical rates varies systematically with the federal funds rate is consistent across all utility models, and holds for data aggregated from both bondholders and nonbondholders. However, though the results from this paper are not conclusive, it is possible that with better data, we may be able to explain part of this discrepancy as resulting from heterogeneity in bond market participation.